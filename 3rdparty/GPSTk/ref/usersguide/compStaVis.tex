%\documentclass{article}
%\usepackage{fancyvrb}
%\usepackage{perltex}
%\usepackage{xcolor}
%\usepackage{listings}
%\usepackage{longtable}
%\usepackage{multirow}
%\RecustomVerbatimEnvironment{Verbatim}{Verbatim}{frame=single}
\definecolor{console}{rgb}{0.95,0.95,0.95}
\lstset{basicstyle=\ttfamily, columns=flexible, backgroundcolor=\color{console}}

\newcommand{\outputsize}{footnotesize}
\newcommand{\application}[1]{\emph{#1}}
\newcommand{\setconsole}{\lstset{basicstyle=\ttfamily, columns=flexible, backgroundcolor=\color{console}}}
\newcommand{\setfileio}{\lstset{basicstyle=\ttfamily, columns=flexible, backgroundcolor=\color{console}}}

\perlnewcommand{\getuse}[1]
{
        my $command = $_[0];
        $command = $command." > temp 2>&1 |head -n 15";
        system("$command");

	my $counter = 0;
	my $done = 0;
	my $output = "";
        open(input,"temp");

        while(my $line = <input>)
	{
		if($done == 0)
		{
			$output = $output.$line if $counter < 15;
			$output = $output." . . .\n" if $counter >= 15;
			$done = 1 if $counter >= 15;
		}

		$counter = $counter + 1;
	}

        close(input);

        return  "\\begin{\\outputsize}\n" . "\\begin{lstlisting}\n" .
                "> ".$_[0]."\n\n" . $output .
                "\\end{lstlisting}\n" . "\\end{\\outputsize}\n";
}

\perlnewcommand{\getrevision}[1]
{
	my $revision = $_[0];
	$revision =~ m/\$LastChangedRevision: ([^\$]*)/;
	return $1;
}

\perlnewcommand{\entry}[4]
{
	my $output = "$_[0] \& $_[1] \& \\multirow{$_[3]}{2.5in}{$_[2]} \\\\";

	my $cnt = 0;
	while($cnt < ($_[3]-1))
	{
		$output = $output." \& \& \\\\ ";
		$cnt = $cnt + 1;
	}
	
	return $output;
}


%\begin{document}
\index{compStaVis!application writeup}
\section{\emph{compStaVis}}
\subsection{Overview}
This application computes station visibility.
\subsection{Usage}
\begin{\outputsize}
\begin{longtable}{lll}
\multicolumn{3}{c}{\application{compStaVis}} \\
\multicolumn{3}{l}{\textbf{Required Arguments}} \\
\entry{Short Arg.}{Long Arg.}{Description}{1}
\entry{-o}{--output-file=ARG}{Name of the output file to write.}{1}
\entry{-n}{--nav=ARG}{Name of navigation file.}{1}
\entry{-c}{--mscfile=ARG}{Name of MS coordinates file.}{1}
& & \\
\multicolumn{3}{l}{\textbf{Optional Arguments}} \\
\entry{Short Arg.}{Long Arg.}{Description}{1}
\entry{-d}{--debug}{Increase debug level.}{1}
\entry{-v}{--verbose}{Increase verbosity.}{1}
\entry{-h}{--help}{Print help usage.}{1}
\entry{-p}{--int=ARG}{Interval in seconds.}{1}
\entry{-e}{--minelv=ARG}{Minimum elevation angle.}{1}
\entry{-t}{--navFileType=ARG}{FALM, FEPH, RNAV, YUMA, SEM, or SP3.}{1}
\entry{-m}{--max-SV=ARG}{Maximum number of SVs tracked simultaneously.}{2}
\entry{-D}{--detail}{Pritn SV count for each interval.}{1}
\entry{-x}{--exclude=ARG}{Exclude station.}{1}
\entry{-i}{--include=ARG}{Include station.}{1}
\entry{-s}{--start-time=TIME}{Start time of evaluation ("m/d/y H:M").}{1}
\entry{-z}{--end-time=TIME}{End time of evaluation ("m/d/y H:M").}{1}
\end{longtable}
\end{\outputsize}

\subsection{Examples}

\subsubsection{Generating station visibility statistics using the SEM almanac from the USCG Navigation Center.}
Same as the previous example, however, the values calculated and the statistics will reflect the number of stations visible to each satellite.
\begin{verbatim}
user@host:~$ compSatVis -ovisout.txt -ncurrent.alm -tYUMA -cstations.msc -e10 -p60 -s"01/13/2008 00:00" -z"01/16/2008 23:59"
\end{verbatim}

%\end{document}
