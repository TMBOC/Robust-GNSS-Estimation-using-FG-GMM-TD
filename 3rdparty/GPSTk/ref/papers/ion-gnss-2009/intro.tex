\section*{Introduction}

The goal of the GPS Toolkit (GPSTk) project is to provide an open
source library and suite of applications to the satellite navigation
community---to free researchers to focus on research, not lower level
coding. In this paper, we explain the organization of the GPSTk project
and its software suite. Because the GPSTk is a collaboration, it grows
over time with new capabilities. Capabilities developed over the last
two years, some of which are still under test, are described.

For exchange of observation data, the GPS community has relied on the
{\bf R}eceiver {\bf IN}dependent {\bf EX}change
format~\cite{rinex1format,rinex2format,rinex211format}.  Since the
GPSTk has been released it has specifically support RINEX version 2
(R2).  To support multi-GNSS receivers and data analysis, RINEX has
evolved from RINEX version 2 to version 3.00, which includes coherent
schemes for multi-GNSS data, as well as greatly enhanced data records
specifically designed for kinematic
applications~\cite{rinex300format}.  Some of the RINEX formats had to
be radically restructured to do so. The GPSTk has developed support
for reading and writing R3 files and mechanisms for storing the
additional data defined by new standard. Existing applications have
been upgraded to use R3.

A key, underlying challenge to truly support R3 is the ability to
integrate observations from multiple Global Navigation Satellite
Systems (GNSSs). Each GNSS defines its own coordinate and time
systems. Reconciling disparate systems could be accomplished on a
case-by-case basis. In the GPSTk, this translates to modifying
processes at the application level. A more seamless solution was
sought for the GPSTk, one that could exist at the library level. The
R3 code transparently provides the translation of coordinate and time
systems during ephemeris evaluation. However this convenience has
implications. The design of the R3 implementation and its implications
are a subject of this paper. Also discussed is how the R3 implementation 
enables the integrated processing of GPS and GLONASS observations.

A topic common to all GNSS processing is clock stability. 
When satellite observations  are used to support time transfer or
orbit determination, clock stability is key factor. The new
GPSTK application \gpstkapp{Clock Tools} provides the GPS and precise time
communities free access to basic clock stability analyses. Current
\gpstkapp{Clock Tools} capabilities include the computation of Allan and Hadamard
stability metrics, along with data parsing, grooming, and plotting. We
present several typical applications.

Fundamental contributions have been made to the \mbox{GPSTk} library as well.
A new auxiliary library has been added that supports precise point
positioning (PPP). Another library has been added to provide customizable
plotting in \LaTeX\ and HTML. 
